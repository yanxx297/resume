%-------------------------
% Resume in Latex
% Author : Sourabh Bajaj
% Website: https://github.com/sb2nov/resume
% License : MIT
%------------------------

\documentclass[letterpaper,11pt]{article}

\usepackage{latexsym}
\usepackage[empty]{fullpage}
\usepackage{titlesec}
\usepackage{marvosym}
\usepackage[usenames,dvipsnames]{color}
\usepackage{verbatim}
\usepackage{enumitem}
\usepackage[pdftex]{hyperref}
\usepackage{fancyhdr}


\pagestyle{fancy}
\fancyhf{} % clear all header and footer fields
\fancyfoot{}
\renewcommand{\headrulewidth}{0pt}
\renewcommand{\footrulewidth}{0pt}

% Adjust margins
\addtolength{\oddsidemargin}{-0.375in}
\addtolength{\evensidemargin}{-0.375in}
\addtolength{\textwidth}{1in}
\addtolength{\topmargin}{-.5in}
\addtolength{\textheight}{1.0in}
\urlstyle{same}
\raggedbottom
\raggedright
\setlength{\tabcolsep}{0in}

% Sections formatting
\titleformat{\section}{
  \vspace{-4pt}\scshape\raggedright\large
}{}{0em}{}[\color{black}\titlerule \vspace{-5pt}]

%-------------------------
% Custom commands
\newcommand{\brief}[2]{
  \item[]\small{
    \textbf{#1}{: #2 \vspace{-2pt}}
  }
}
\newcommand{\myitem}[1]{
  \item[-]\small{
    { #1 \vspace{-2pt}}
  }
}
\newcommand{\education}[6]{
  \vspace{-1pt}\item[]
    \begin{tabular*}{0.97\textwidth}{l@{\extracolsep{\fill}}r}
      \textbf{#1} & {\small #2} \\
      \textit{\small#3} & \textit{\small #4} \\
      \ifx\hfuzz#5#6\hfuzz
      \else
      \textit{\small#5} & \textit{\small #6} \\
      \fi      
    \end{tabular*}\vspace{-5pt}
}
\newcommand{\project}[3]{
  \vspace{-1pt}\item[]
  \begin{tabular*}{0.97\textwidth}{l@{\extracolsep{\fill}}r}
  \textbf{\small#1} & {\small#2}\\
  \end{tabular*}
      {\small#3}\vspace{-5pt}
}
\newcommand{\bib}[4]{
  \vspace{-1pt}\item[]
      {\small#2},
      \ifx\hfuzz#4\hfuzz
      \textbf{\small``#1,''}      
      \else
      \href{#4}{\small``#1,''}
      \fi      
      \small{#3}
    \vspace{-2pt}
}
\newcommand{\content}{\begin{itemize}[leftmargin=0px]}
\newcommand{\contentend}{\end{itemize}}
\newcommand{\mylist}{\begin{itemize}[leftmargin=25px,rightmargin=25px]}
\newcommand{\mylistend}{\end{itemize}\vspace{-5pt}}


%-------------------------------------------
%%%%%%  CV STARTS HERE  %%%%%%%%%%%%%%%%%%%%%%%%%%%%
\begin{document}
%----------HEADING-----------------
\begin{tabular*}{\textwidth}{l@{\extracolsep{\fill}}r}
  \textbf{{\Large Qiuchen Yan}} & Email : 
  {yanxx297@umn.edu}\\
  \href{http://www-users.cs.umn.edu/~yanxx297/}
  {http://www-users.cs.umn.edu/$\sim$yanxx297/}
  & Mobile : +1-651-235-4138 \\
  \href{https://github.com/yanxx297}{https://github.com/yanxx297}
\end{tabular*}

%-----------EDUCATION-----------------
\section{Education}
  \content
    \education
      {University of Minnesota, Twin Cities}{Minneapolis, MN}
      {Ph.D. in Computer Science;  GPA: 3.60}
      {May 2014 -- 2020 (anticipated)}
      {Master of Science in Computer Science;  GPA: 3.625}
      {Sep. 2012 -- May 2014}          
    \education
      {Shandong University of Science and Technology}{Qingdao, China}
      {Bachelor of Engineering in Computer Science;  GPA: 3.65}
      {Sep. 2008 -- July 2012}{}{}
  \contentend

%-----------PROJECTS-----------------
\section{Research Projects}
  \content  	
    \project
      {Fast \& Automatic Emulator Testing System}{2015 -- 2018}{}
    \mylist
        \myitem
          {Speed up an automatic emulator testing tool by 200 times by designing and
          implementing a novel approach to generate test cases.}
        \myitem{
          Implemented a x86 assembly test cases generator in Python based on previous work.
          Also modified other components of the testing system written in C++.
        }
      \mylistend         
    \project
      {Loop Summarization for Symbolic Execution}{2014 -- 2015, 2018 -- present}{}  
      \mylist
        \myitem{
          As a countermeasure of the path explosion problem, 
          design a extended version of a trace-based loop summarization algorithm\cite{Godefroid2011}
          and implement it on FuzzBALL, a symbolic execution platform.
        }   
        \myitem{
          Currently consists of approximately 1300 lines of OCaml code.
        }
      \mylistend          
    \project
      {Binary Level Type Inference}{2013 -- 2014}{}
      \mylist
        \myitem
          {Design a static type inference tool that can infer 
          the signedness of variables in binaries with 96\% true positive.}        
        \myitem{
          Build this tool on top of Vine and libdwarf using C++.
        }
      \mylistend    
  \contentend

%
%--------PUBLICATION------------
\section{Publication}
	\content
    \bib
      {Fast PokeEMU: Scaling Generated Instruction Tests Using 
      Aggregation and State Chaining}
      {\textbf{Qiuchen Yan}, Stephen McCamant}
      {The 14th ACM SIGPLAN/SIGOPS International Conference on Virtual Execution Environments (VEE'18)}
      {http://www-users.cs.umn.edu/~yanxx297/vee18-fast-pokeemu.pdf}
    \bib
      {Fast PokeEMU: Scaling Generated Instruction Tests Using 
      Aggregation and State Chaining}
      {\textbf{Qiuchen Yan}, Stephen McCamant}
      {Poster}
      {http://www-users.cs.umn.edu/~yanxx297/posterFastPokeEMU.pdf}     
    \bib
      {Conservative Signed/Unsigned Type Inference for Binaries
      using Minimum Cut}
      {\textbf{Qiuchen Yan}, Stephen McCamant}
      {Technical report}
      {https://www.cs.umn.edu/research/technical_reports/view/14-006}             
  \contentend         

%
%--------EXPERIENCE------------
\section{Experience}
  \content
  \project
      {Graduate Research Asistant, University of Minnesota}{2014 -- present}
      {Work with Stephen McCamant on several research projects. 
      Collaborate with Pen-Chung Yew's dynamic binary translation group on the emulator testing project.}    
    \project
      {DARPA Cyber Grand Challenge}{2014 -- 2015}
      {Contributed bug checking code for FuzzBOMB group in CGC Qualification Event.}  
  \contentend
  
  
%
%--------PROGRAMMING SKILLS------------
\section{Skills}	
 \content
  \brief
    {Programming Languages}
    {C/C++, Python, OCaml, X86 assembly, Javascript, PHP, SQL} 
  \brief
    {Systems \& Tools}
    {Linux, Xed (Intel Pin), DWARF, Vine, FuzzBALL}
  \brief
    {Version Control System}{Git}
 \contentend

%
%--------RELATED COURSEWORK------------
\section{Coursework}
  \content
    \brief
      {Introduction to Computer Security}
      {A breadth of knowledge about software security and network security}   
    \brief
      {Modern Cryptography}
      {Introduction to widely used cryptography theories and algorithms}
    \brief
      {Machine Learning}
      {Introduction to machine learning}      
    \brief
      {Security and Privacy in Computing}
      {A seminar discussing recent papers about security, privacy and cryptography}
  \contentend  

%-------------------------------------------
\section{Academic Projects}
  \content  	
    \project
      {Reproduce the Lucky Thirteen attack}{2014}
      {Implemented a timing side channel attack to the TLS protocol. 
    Course project.}
    \project
      {Sybil attack study}{2014}
      {Surveyed about the Sybil attack in online social network and its state-of-art defence approach
      and collected data from real world sybil communities in sina weibo. Course project.}
    \project
      {Encrypted address book}{2012}
      {An Android address book app that can send encrypted contact info via text message. 
      Bachelor final project.}
  \contentend

%-------------------------------------------
\section{Service}
\mylist
\myitem{Contribut code to FuzzBALL, an open source symbolic execution tool.}
\myitem{Gave guest lectures on security related courses at the University of Minnesota.}
\mylistend

\bibliographystyle{unsrt}
\bibliography{ref}
\end{document}
